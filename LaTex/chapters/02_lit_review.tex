%The existing literature on the use of repression by a state have mostly focused on the impact of  of domestic factors such as institutions and economic growth \citep{davenport_state_2007-1}.
Existing literature on repressions has focused mostly on their consequences and legacies \citep{rozenas_political_2017, lupu_legacy_2017, zhukov_stalins_2018}. As far as the strategic use of repressions by the state is studied, it is usually in relation to domestic factors such as institutions and economic shocks \citep{davenport_state_2007-1, svolik_politics_2012, greitens_dictators_2016, blaydes_state_2018} with less attention being given to external forces.
\citet{greitens_dictators_2016} links the severity of repression to the threat from dictator's inner circle. A dictator who fears that he would be deposed in a coup rather than in popular uprising will fragment their coercive apparatus in order to weaken the power of a potential challenger from within. The weakened secret police will be, according to Greitens, more likely to use violence since it fails to identify the transgressors and cannot effectively deter dissent. 
Other scholars see repression as a substitute for co-option \citep{wintrobe_political_1998}. Instead relying on the threat of persecution, an authoritarian ruler might buy the loyalty of the population by distributing rents to the supporters of the regime. 
\citet{blaydes_state_2018} describes how 
illustrates on a case of Iraq under Hussein the relationship between repression and economic condition. Whereas in the period of high oil prices the regime could use the rents to buy the support of the public, this became 

Moreover, \citet{blaydes_state_2018} explains the different level of repression. She argues that the nature of repression  depends on the legibility of the ethnic group to the state \footnote{By legibility we mean ability of the state to identify individuals in a given population and gather information on them}
(i.e. ability to gather information on its population). Since the state coercive institutions cannot reliably transgressors in  culturally and logistically (and thus less legible population, they will more likely resort to collective punishment (based on ethnicity, religion or community membership) . The logic behind this is that the members of the community will police its members to avoid collective punishment. 
%argues that repression can be a subsitute for economic rents.  

Our research also contributes to the literature studying factors that influence position of a state towards its ethnic minorities and under what conditions conflict is likely to emerge. Size and and distribution of ethnic groups have been emphasized. Several scholars pointed out that
states with large number of ethnic groups are more likely to violently repress calls for autonomy or secession to discourage other ethnic minorities from making similar demands in the future \citep{evera_hypotheses_1994, toft_geography_2005,walter_reputation_2009}. 
% On the other hand, binational countries such as former Czechoslovakia  
Furthermore,  \citet{toft_geography_2005} argues that geographically concentrated groups tend to view their ethnic homeland as indivisible and non-negotiable issue which increases the likelihood of violent conflict. However, these approaches fail to explain changes in state's attitudes to minorities over short periods of time when the size and distribution of ethnic groups remains roughly constant. 
% možná tam hoď Votes and Violence od Wiliknsona

More recently, the role of international actors have received greater attention.  \citet{butt_secession_2017}  
%\citet{svolik_politics_2012} sees repression as one of the tools of the authorian regime to keep conrol. Co-option and repression. Whereas the co-option is costly, repression requires greater power to the military. This might theaten the dictator's rule. However, this does not address different levels of repressions applied to different ethnic groups. 

As was mentioned, \citet{mylonas_politics_2013} proposes a theory how of geopolitical relations influence the attitude of a state towards its minorities. He also tests his theory with data on the post-World War I Balkans where the nation-building policies (categorized into 3 groups: accommodation, assimilation and exclusion)  toward  90 ethnic groups are a dependent variable and information on their support by external powers is an explanatory variables (together with other control variables). However, the results of the cross-sectional regression, used in the study, might easily be biased due to omitted variables or reverse causality and we believe that our approach offers cleaner identification.

%According to \citet{blaydes_state_2018}, a state will resort to collective punishment (based on ethnicity, religion or community membership) if it faces environment with highly asymmetric information in which it cannot identify the likely transgressors. The logic behind this is that the members of the community will police its members to avoid collective punishment. 

Our main contribution is empirical. Most studies test the theories by qualitative comparison of selected cases. To the 

\citet{mcnamee_demographic_nodate} is methodologically and thematically closet study to ours. They analyze how the 1958 split in Soviet-China relations affected the demographic composition of the population in the Soviet-Chinese border regions.
Using difference-indifference strategy, they find that, after the split,  both states supported expulsions  of the minority group and sponsored immigration of the majority group but only in border regions without significant natural boundary (e.g. high mountains). They conclude that the states use demographic engineering as a way to protect their vulnerable border against a hostile power. 


%Hod tam koyamu, Zhurkavskayu a doten-snitken