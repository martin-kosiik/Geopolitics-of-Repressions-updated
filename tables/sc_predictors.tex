\begin{table}[!h]

\caption{\label{tab:sc_predictors}Pre-treatment characteristics of ethnic groups in the USSR}
\centering
\begin{threeparttable}
\begin{tabular}{lrrrr}
\toprule
Ethnic group & Total population & Ling. similarity to Russian & Urbanization rate & Ind. state\\
\midrule
Altai & 39 062 & 0 & 0.30 & 0\\
Armenian & 1 567 568 & 1 & 35.45 & 0\\
Balkar & 33 307 & 0 & 1.23 & 0\\
Bashkir & 713 693 & 0 & 2.12 & 0\\
Belorussian & 4 738 923 & 4 & 10.32 & 0\\
Bulgarian & 111 296 & 3 & 6.26 & 0\\
Buryat & 237 501 & 0 & 1.05 & 0\\
Estonian & 154 666 & 0 & 23.00 & 1\\
Finnish & 134 701 & 0 & 10.55 & 1\\
Georgian & 1 821 184 & 0 & 16.93 & 0\\
German & 1 238 549 & 1 & 14.92 & 1\\
Greek & 213 765 & 1 & 21.21 & 1\\
Hungarian & 5 476 & 0 & 63.33 & 1\\
Chechen & 318 522 & 0 & 0.98 & 0\\
Chinese & 10 247 & 0 & 64.87 & 1\\
Chuvash & 1 117 419 & 0 & 1.60 & 0\\
Japanese & 93 & 0 & 76.34 & 1\\
Jewish & 2 599 973 & 1 & 82.43 & 0\\
Kabardian & 139 925 & 0 & 1.27 & 0\\
Kalmyk & 129 321 & 0 & 1.29 & 0\\
Karelian & 248 120 & 0 & 2.91 & 0\\
Kazakh & 3 968 289 & 0 & 2.18 & 0\\
Khakas & 45 608 & 0 & 1.08 & 0\\
Komi & 375 871 & 0 & 2.56 & 0\\
Korean & 86 999 & 0 & 10.52 & 0\\
Latvian & 141 703 & 2 & 42.31 & 1\\
Lithuanian & 41 463 & 2 & 63.16 & 1\\
Mari & 428 192 & 0 & 0.84 & 0\\
Moldovan & 278 905 & 1 & 4.86 & 0\\
Mordvin & 1 340 415 & 0 & 2.19 & 0\\
Ossetian & 272 272 & 1 & 7.86 & 0\\
Polish & 782 334 & 3 & 32.75 & 1\\
Russian & 77 791 124 & 5 & 21.32 & 1\\
Tatar & 2 916 536 & 0 & 15.48 & 0\\
Udmurt & 504 187 & 0 & 1.21 & 0\\
Ukrainian & 31 194 976 & 4 & 10.54 & 0\\
Uzbek & 3 904 622 & 0 & 18.66 & 0\\
Yakut & 240 709 & 0 & 2.20 & 0\\
\bottomrule
\end{tabular}
\begin{tablenotes}
\item \textit{Note: } 
\item Total population and urbanization rate of the ethnic group in the USSR is taken from 1926 census. The linguistic similarity to Russian is measured by the number of common nodes in the language tree (cladistic similarity). Independent state equals one if the ethnic group was a core group in an independent country that existed in the interwar period.
\end{tablenotes}
\end{threeparttable}
\end{table}