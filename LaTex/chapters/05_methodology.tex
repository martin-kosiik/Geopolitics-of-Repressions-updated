\section{Difference-in-differences}
Our main specification is the dynamic difference-in-differences model:
\begin{equation}
 \log\left(1 + y_{it}\right) = \sum_{k= \text{1928Q1}}^{\text{1939Q4}} \beta_k \, \text{German}_{i} \cdot \text{YearQuart}_{t}^k + \lambda_t + a_i +  E_i \cdot t \:  + E_i \cdot t^2 \:    + u_{it}
 \label{eq:dynamic_did}
\end{equation}

 %+ \sum_{j= 1}^5 \omega_j \, german_{it} \cdot post_{it - k}    

where $y_{it}$ is number of arrests of people with ethnicity $i$ in year-quarter $t$, $\lambda$ is year fixed effect, $a$ is ethnicity fixed effect (both captured by respective dummy variables) and  $\text{YearQuart}_{t}^k$ are dummy variables that equals 1 if its year-quarter $k$ equals to $t$ and 0 otherwise (except for 1939Q4 which is equal to 1 after the fourth quarter of 1939 and zero otherwise) . The coefficients of interest are $\beta_k$. 
Prior to second quarter of 1933 they capture the pre-treatment trends the lead (anticipatory) effects  used to test if pre-treatment trends are parallel. After second quarter of 1933 they the dynamic lagged effects.
%The 5 coefficients $\omega_j$ capture the potential lagged effects (extending from 1934 to 1938), whereas the 3 coefficients capture the lead (anticipatory) effects (from 1930 to 1932) used to test pre-treatment parallel trends. 
The $ E_i \cdot t$ and $ E_i \cdot t^2$  term capture the ethnicity specific quadratic time trends. The inclusion of this term should not significantly  change the coefficients, unless the results are driven by spurious correlation (\citealt{angrist_mostly_2009}). 

 We apply logarithmic transformation on $y_{it}$ since it better fits the data (more in the results below).  We use $\log\left(1 + y_{it}\right)$ because some observations (although not many) have $y = 0$. As discussed in \citet[p. 193]{wooldridge_introductory_2015},  the percentage change interpretation is usually  closely preserved (except for changes beginning at 0 which are not of interest to us).   

Our identifying assumption is that the number of arrest of Germans after 1933 would go in parallel to arrests of other minorities in the absence shock to German-Soviet relations conditional on our control variables (mainly the ethnicity specific time trends). Although we cannot test this assumption, we can test whether the trends were parallel prior to 1933 (pre-treatment) which could increase our confidence that they were parallel after 1933 too. This can be testing if the coefficients $\beta_k$ on the lead effects are significantly different from zero.  

As \citet{bertrand_how_2004} show, the usual standard errors  are downward-biased for most DiD regressions since they do not account for the serial correlation within the units of interests (states, countries etc.). A common solution to this problem is to estimate standard errors using robust covariance matrix that allows for clustering (i.e. cluster-robust standard errors). However for small number of groups (generally less than 40), the cluster-robust standard errors are downward-biased and not reliable. \citet[chapter 8]{angrist_mostly_2009} suggest taking the maximum of cluster-robust as a simple rule of thumb to avoid gross misjudgements in precision. More rigorous solutions are cluster bootstrapping \citep{cameron_bootstrap-based_2008, cameron_practitioners_2015} and  using $t$-distribution with $G- K$ degrees of freedom (where $G$ is number of clusters and $K$ number of parameters) rather than the standard Normal distribution \citep{mccaffrey_bias_2002, imbens_robust_2016}.
Since we have small number of groups we use the generalization of \citet{mccaffrey_bias_2002} correction to models with arbitrary sets of fixed effects by \citet{pustejovsky_small-sample_2018}.

\subsection{Results}
The estimated coefficients  $\beta_k$ from the specification \ref{eq:dynamic_did} are plotted in the figure \ref{fig:did_effets}. The coefficients between the years 1933 and 1939 (when the relations between Germany and Soviet Union were hostile) mostly range  between 1 and 3 and all except one are statistically significant at 5\% level. The rise of Nazism thus based on these estimated increased the arrests of Germans by the NKVD in the USSR by about 2\%.  

However, the pre-1933 coefficients give us some reason to doubt the
validity of our model. Three of them are significantly different from 0 at 5\% level and others are very close to being significant. 
This provide some evidence that the pre-treatment trends for German minority were not parallel with trends for other minorities. We can thus suspect that the post-treatment trends were not parallel either which
would violate the basic identifying assumption of
difference-in-differences. To address this concern, we apply the synthetic control method which can be valid even in the absence of  parallel trends. 
%We can see that all coefficients for years 1930 to 1932 are statistically insignificant which means that the pre-treatment trends in arrests of German minority were likely parallel to the pre-treatment trends of other minorities which gives us greater  confidence in the validity of our identification strategy. 

%The coefficients on all other years are insignificant as well. Only for 1934 (one year lag) is the estimate  significant at 10\% level ($p$-value of 0.08). Since this not reaches even the traditional 5\% significance threshold we are inclined to not reject the null hypothesis or at least to conclude that evidence in favor of the alternative hypothesis (more repressions of Germans due to rise of Hitler) is quite weak.  Furthermore, as we show below the alternative specifications do not increase the significance of the coefficients.
\begin{figure}[h]
\centering
\includegraphics[width=\textwidth]{plots/effects/quarterly/until_pact/pointrange_robust.pdf}
\caption{Estimates of coefficients $\beta_k$ on the interactions of $\text{German}_{i} \cdot \text{YearQuart}_{t}^k$}
\label{fig:did_effets}
\end{figure}
%We perform several robustness checks to asses sensitivity of the results to different specifications. First, in our main model (column (1) of table \ref{dif_table}), we included all observations in years 1923 to 1958. But the relationship between Germany and Soviet Union were somewhat more complicated after the World War II. We thus re-estimate the model with only the data from 1923 to 1945. The results (in column (2)) change only little and does not alter our previous conclusions. Second, when we omit the ethnicity specific linear time trends in column (3), we see again that the coefficients are very similar to the original model. Finally, we estimate a specification with number of arrests as a dependent variable (without logarithmic transformation). We can see that this model (shown in column (4)) fits the data rather poorly with  $R^2$ only of 0.428 (compared to 0.890 in the logarithmic specification). 

\section{Synthetic Control Method}
However, the parrarel trends assumption can sometiemes be violated. These issues can be adressed by sythetic control method
\citep{abadie_economic_2003, abadie_synthetic_2010}

Let $Y_{it}$ be the outcome of a unit $i$ at time $t$ with $i = 1$ being the treated group.  We
denote $D_{1t}$ as the treatment dummy, i.e. variable that equals 1 if $i = 1$ and $T > T_0$ and 0 otherwise (with $T_0$ being the start of the treatment). 
Let be $Y_{1t}^N$ be a counterfactual outcome for the treated unit in the absence of treatment. The effect of treatment a time $t$, $\alpha_{1t}$ is assumed to be given as 
\begin{equation}
    Y_{1t} = Y_{1t}^N + \alpha_{1t} \, D_{1t}
\end{equation}
Furthermore, the synthetic control method assumes that $Y_{1t}^N$  can be expressed by the following factor model:
\begin{equation}
   Y_{1t}^N = \delta_t + \boldsymbol{\theta}_t \boldsymbol{Z}_i +
   \boldsymbol{\lambda}_t \boldsymbol{\mu}_i + \epsilon_{it}
\end{equation}
where is $\delta_t$ an unknown common factor with constant factor
loadings across units, $\boldsymbol{Z}_i$ is a
$(1 \times r)$ vector of observed time-invariant covariates (unaffected by the treatment),  $\boldsymbol{\theta}_t$ is a $(1 \times r)$ vector of
unknown parameters, $\boldsymbol{\lambda}_t$ is a $(1 \times F)$ vector of unobserved time-varying factors, $\boldsymbol{\mu}_i$ is an $(F \times 1)$ vector of unknown factor loadings
and $\epsilon_{it}$ is the error term with zero mean.

Notice that for constant  $\boldsymbol{\lambda}_t$ for all $t$ we get the traditional  difference-in-differences model. Unlike difference-in-differences,  the synthetic control method  allows for unit-specific time trends as long as they can be captured by the factor model. 

The synthetic control is  constructed as a convex combination of available comparison units (in our case other minorities in the USSR) that most closely resembles the pre-treatment characteristics of the treated group
 (or more precisely, for which the average of its factor loadings $\boldsymbol{\mu}_i$ match the factor loadings of the treated unit  $\boldsymbol{\mu}_1$). 
 More formally we choose weights $W = (w_2, \dots, w_J, w_{J+1})$ subject to $w_j \geq 0$ for $j = 1, \dots, J, J + 1$ and $w_2 +  \dots + w_J + w_{J+1} = 1$ that minimize $\left\Vert X_1 - X_0 W \right\Vert$ where $X_1 = (Z_1, Y_1^{K_1}, \dots, Y_1^{K_L})$ is a $(k \times 1)$ vector of pre-treatment characteristics of the treated unit and $k = r + L$ and $Y^{K_l}$ are combinations of
pre-treatment outcomes (analogously for $X_0$). The effect of the treatment at time $t$, $\alpha_{1t}$, is then estimated as a difference between the outcome for synthetic control and the treated unit, i.e.:
\begin{equation*}
  \hat\alpha_{1t}  = Y_{1t} - \sum_{k = 2}^{J + 1} w_j^* Y_{kt}
\end{equation*}

\subsection{Results}
We implemented the synthetic control method in R software using the MSCMT package \citep{becker_fast_2018}. The calculated optimal weights $W$ of ethnic groups in the synthetic German minority are provided in the table \ref{tab:sc_weights} (ethnic groups with zero weight are not shown). The highest contribution in the synthetic German minority have the Ossetians, Tatars and Pols with weights 0.39, 0.23 and 0.14 respectively. The Greek, Kabaradin, Chechen, Lithuanian and Ukrainian minorities are also represented in the synthetic control although only with very small weights. 
\begin{table}[t]

\caption{\label{tab:sc_weights}Synthetic German minority weights}
\centering
\begin{tabular}{lr}
\toprule
Ethnic group & $W$-Weight\\
\midrule
Russian & 0.36\\
Greek & 0.28\\
Finnish & 0.17\\
Lithuanian & 0.08\\
Khakas & 0.05\\
Yakut & 0.05\\
Bulgarian & 0.01\\
\bottomrule
\end{tabular}
\end{table}


Figure \ref{fig_sc_comp_plot} shows the trends in arrests for the German minority and its synthetic control. The synthetic German minority tracks the actual values fairly well except for two large negative shocks to the actual arrests in 1931 and 1932 which the synthetic control does not capture. The trends start to diverge in the second quarter of 1933 with the actual arrests of Germans holding steady but decreasing for synthetic control. 
The gap between the trends for the actual German minority and its synthetic control (shown in figure \ref{fig:sc_placebo_gaps_all}) keeps within the range of 1.25 to 2.5 for most of the post-1933 period. This implies that the rise of Hitler led to about  2\% increase in the arrests of Germans by the NKVD  in the period from 1933 to 1939. This is very similar to the estimates obtained using difference-in-differences. 
%The figure shows increase between 1937 and 1939 (the period of the Great Terror).  
\begin{figure}[h]
\centering
\includegraphics[width=\textwidth]{plots/synthetic_control/until_pact/comparison_plot.pdf}
\caption{Comparison plot}
\label{fig_sc_comp_plot}
\end{figure}


%\subsubsection{Inference}
The synthetic control method, however, by itself does not provide us with any measure of uncertainty of significance. This has been addressed by performing placebo tests \citep{abadie_synthetic_2010}. Synthetic control method is applied iteratively to every ethnicity in the donor pool
%with the same treatment period
as if they were treated. By comparing the gaps from these placebo tests with the gap for the German minority, we can assess the
significance of our results.
Large gaps for arrests of Germans relative to other ethnic groups would suggest that the results are significant since these results would be less likely if there were no treatment effect . 
%The estimated  gaps between the synthetic control and the actual data for every ethnic group is shown in figure \ref{fig:sc_placebo_gaps}. 

Figure \ref{fig:sc_placebo_gaps_all} shows gaps between the synthetic control and the actual trends for Germans together with placebo gaps for all 17 other ethnic groups. The post-treatment gap for German minority is relatively large although not the highest. The figure also highlights that for some ethnic groups the pre-treatment gaps are large too. This indicates that synthetic control of these ethnic groups does not capture the actual pre-treatment trends well. As \citet{abadie_synthetic_2010} note, placebo synthetic controls  with poor pre-treatment fit do not provide good comparison
%are not very helpful for
for estimating rareness of large  post-treatment gap for a treatment with a good pre-treatment fit. They thus recommend excluding excluding placebo groups with substantially higher pre-treatment mean squared prediction error  (MSPE)
%(the average of squared differences between the value of the outcome for synthetic and actual) 
%relatively to the treated group. 
%In our case the pre-treatment MSPE of the German minority is fairly small (0.52). 
%If we exclude minorities with the pre-treatment MSPE
%If we choose the cutoff for exclusion as the MSPE 

%Morever, the pre-treatment MSPE is quite small (0.52) which means that the synthetic German minority captures the actual pre-treatment trends relatively well. However, the pre-treatment MSPE of a few minorities is much larger indicating that no combination of donor ethnic groups fits well their time series. As \citet{abadie_synthetic_2010} note,  if there is poor fit of the synthetic control to the actual trends in the pre-treatment period then its post-treatment gap does not provide good comparison to the well fit ethnic groups. 
%large post-treatment gap would not indicate the presence of an effect but  
Following \citet{abadie_synthetic_2010}  we therefore exclude ethnic groups whose pre-treatment MSPE is 5 times higher then the same measure for German minority. This removes 4 ethnic groups with the worst pre-treatment fit (Tatars, ...). The resulting plot is shown in the figure \ref{fig:sc_placebo_gaps_all_5_times}. The post-1933 gaps in German arrests now stand out more clearly. 

\begin{figure}[hbtp] 
\begin{subfigure}{\textwidth}
\includegraphics[width=0.9\linewidth]{plots/synthetic_control/until_pact/placebo_highlight_all.pdf}
\caption{All ethnic groups}
\label{fig:sc_placebo_gaps_all}
\end{subfigure}
\begin{subfigure}{\textwidth}
\includegraphics[width=0.9\linewidth]{plots/synthetic_control/until_pact/placebo_highlight_mspe_5_lower.pdf}
\caption{Ethnic groups with pre-treatment MSPE higher than 5 times the MSPE of Germany excluded}
\label{fig:sc_placebo_gaps_all_5_times}
\end{subfigure}
\caption{Gaps between synthetic control and actual values for placebo tests}
\label{fig:sc_placebo_gaps}
\end{figure}

Nevertheless, the choice of any level of the cutoff of pre-treatment MSPE %for exclusion of 
is somewhat arbitrary. Alternative way to asses significance of results may be to compare the ratios of  post/pre-treatment MSPE.  The values of these ratios for all ethnic groups are displayed in the figure \ref{fig:sc_mspe_ratios}. Post/pre-treatment MSPE ratio for the German minority is by far the highest. The probability of German minority having the highest ratio of all under the null hypothesis of zero treatment effect is 1/17 ($\approx 0.06$). 

%This is shown in  the figure \ref{fig_sc_mspe_hist}
\begin{figure}[h]
\centering
\includegraphics[width=\textwidth]{plots/synthetic_control/mspe_ratios.pdf}
\caption{Ratios of post-treatment MSPE to pre-treatment MSPE}
\label{fig:sc_mspe_ratios}
\end{figure}
%\subsection{Generalized Synthetic Control Method}
%\citet{xu_generalized_2017}