\subsection{Difference-in-differences}
We start by presenting results for the data with imputed ethnicity (adjusted with full matrix correction from equation \ref{eq:full_matrix_adj} and date of arrest since we consider them to be the most credible. However, we also consider how the results change if do not impute or apply different adjustment further in the paper. 

 The estimated coefficients  $\beta_k$ from the dynamic difference-in-differences model (\ref{eq:dynamic_did}) are plotted in the figure \ref{fig:did_effets}. 
 We can draw several conclusions from the results. 
 First, the coefficients from 1933 to 1938 are not significantly different from zero which implies that hostility between Germany and the Soviet Union in this period does not seem to impact the repressions of Germans contrary to the theoretical predictions. Second, the political arrests of Germans start to rise in 1939 which is surprising given that  Molotov-Ribbentrop pact guaranteeing neutrality Germany and the USSR was signed that year. The repressions then peak  with start of the war at 1941 (although there is some slight drop from 1942 to 1944). Finally, the effects of war on the repressions appear to be highly persistent as there is no decline in the arrests after 1945. 
 %There even
 
 \begin{figure}[h]
\centering
\caption{Estimates of $\beta_k$ from the Specification \ref{eq:dynamic_did}}
\includegraphics[width=\textwidth]{plots/effects/ethnicity_imputation/annual/point_range_robust_cr2_date_imp_full_years.pdf}
\begin{minipage}{0.92\textwidth}
%\includegraphics[width=\linewidth]{Gaus.pdf}
%\rule{\linewidth}{10em}
\footnotesize
\emph{Notes:} Ethnicity and date of arrest were imputed.  Full matrix adjustment was applied on ethnic group imputations. All 38 ethnic groups are included.  Standard errors are based on cluster robust estimator by \citet{pustejovsky_small-sample_2018}. Error bars show 95\% confidence intervals. 
\end{minipage}
\label{fig:did_effets}
\end{figure}

 
 The results from the time window specification (equation \ref{eq:simple_did}) shown in the figure \ref{fig:did_effets_time_window} tell the same story.  The coefficients in the model imply that in the period of Molotov-Ribbentrop pact and war, the arrests of Germans were about 2\% higher than the arrest other ethnic groups. Surprisingly, the estimated effect in the post-war period is higher still at  3.6\%.
 
However, the pre-1933 coefficients in the figure \ref{fig:did_effets} give us some reason to doubt the
validity of our model. Even though they are small in size relative to the post-1939 coefficient, they are all  significantly different from 0 at 5\% level.
%and others are very close to being significant. 
This provide some evidence that the pre-treatment trends for German minority were not parallel with trends for other ethnic groups. We can thus suspect that the post-treatment trends were not parallel either which
would violate the basic identifying assumption of
difference-in-differences.
%The deviations from parallel trends in the pre-treatment period are relatively small possibly indicating that the bias might not be as large. 
We address this problem by applying the synthetic control method in the next section. 
%Bearing in mind these potential problems, we proceed 
 
Another potential issue is that some ethnic groups changed their treatment status in this period in various complicated ways. For example, Poland was invaded by the USSR and Germany but it took  only about month until the Polish forces were defeated. The Baltic states were annexed and incorporated into the Soviet Union in June 1940 only to be invaded by Germany year later. 
It is difficult to decide what treatment status should we assign in these cases. 
Moreover, given the scale and complexity of World War 2, the USSR experienced  some kind of change in geopolitical relations with almost every country in its neighborhood.  

To address this issue, we exclude every ethnicity which constituted a core group in some independent state in the interwar period from the dataset (except for Germans, of course) and reestimate the model. This criterion removes 10 ethnic groups from the dataset. The full list of them is provided in the table \ref{tab:sc_predictors}. 
%Bearing in mind these potential problems, we continue by considering different sets of control groups to better understand what drives the results. 
 
 %The coefficients between the years 1933 and 1939 (when the relations between Germany and Soviet Union were hostile) mostly range  between 1 and 3 and all except one are statistically significant at 5\% level. The rise of Nazism thus based on these estimated increased the arrests of Germans by the NKVD in the USSR by about 2\%.  

%However, the pre-1933 coefficients give us some reason to doubt the
%validity of our model. Three of them are significantly different from 0 at 5\% level and others are very close to being significant. 
%This provide some evidence that the pre-treatment trends for German minority were not parallel with trends for other minorities. We can thus suspect that the post-treatment trends were not parallel either which would violate the basic identifying assumption of difference-in-differences. To address this concern, we apply the synthetic control method which can be valid even in the absence of  parallel trends. 
%We can see that all coefficients for years 1930 to 1932 are statistically insignificant which means that the pre-treatment trends in arrests of German minority were likely parallel to the pre-treatment trends of other minorities which gives us greater  confidence in the validity of our identification strategy. 

%The coefficients on all other years are insignificant as well. Only for 1934 (one year lag) is the estimate  significant at 10\% level ($p$-value of 0.08). Since this not reaches even the traditional 5\% significance threshold we are inclined to not reject the null hypothesis or at least to conclude that evidence in favor of the alternative hypothesis (more repressions of Germans due to rise of Hitler) is quite weak.  Furthermore, as we show below the alternative specifications do not increase the significance of the coefficients.


\begin{figure}[h]
\centering
\caption{Estimated Coefficients for Specified Time Windows from the Specification \ref{eq:simple_did} }
\includegraphics[width=\textwidth]{plots/effects/ethnicity_imputation/annual/time_window_pre_treatment_date_imp.pdf}
\begin{minipage}{0.92\textwidth}
%\includegraphics[width=\linewidth]{Gaus.pdf}
%\rule{\linewidth}{10em}
\footnotesize
\emph{Notes:} Ethnicity and date of arrest were imputed.  Full matrix adjustment was applied on ethnic group imputations. All 38 ethnic groups are included. Standard errors are based on cluster robust estimator by \citet{pustejovsky_small-sample_2018}. Error bars show 95\% confidence intervals. 
\end{minipage}
\label{fig:did_effets_time_window}
\end{figure}



%We perform several robustness checks to asses sensitivity of the results to different specifications. First, in our main model (column (1) of table \ref{dif_table}), we included all observations in years 1923 to 1958. But the relationship between Germany and Soviet Union were somewhat more complicated after the World War II. We thus re-estimate the model with only the data from 1923 to 1945. The results (in column (2)) change only little and does not alter our previous conclusions. Second, when we omit the ethnicity specific linear time trends in column (3), we see again that the coefficients are very similar to the original model. Finally, we estimate a specification with number of arrests as a dependent variable (without logarithmic transformation). We can see that this model (shown in column (4)) fits the data rather poorly with  $R^2$ only of 0.428 (compared to 0.890 in the logarithmic specification). 

%We perform several robustness checks to asses sensitivity of the results to different specifications. First, in our main model (column (1) of table \ref{dif_table}), we included all observations in years 1923 to 1958. But the relationship between Germany and Soviet Union were somewhat more complicated after the World War II. We thus re-estimate the model with only the data from 1923 to 1945. The results (in column (2)) change only little and does not alter our previous conclusions. Second, when we omit the ethnicity specific linear time trends in column (3), we see again that the coefficients are very similar to the original model. Finally, we estimate a specification with number of arrests as a dependent variable (without logarithmic transformation). We can see that this model (shown in column (4)) fits the data rather poorly with  $R^2$ only of 0.428 (compared to 0.890 in the logarithmic specification). 


\subsection{Synthetic Control Method}
We implemented the synthetic control method in R  using the MSCMT package \citep{becker_fast_2018}. The calculated optimal weights $W$ of ethnic groups in the synthetic German minority are provided in the table \ref{tab:sc_weights} (ethnic groups with zero weight are not shown). The highest contribution in the synthetic German minority have the Ossetians, Tatars and Pols with weights 0.39, 0.23 and 0.14 respectively. The Greek, Kabaradin, Chechen, Lithuanian and Ukrainian minorities are also represented in the synthetic control although only with very small weights. 
\begin{table}[t]

\caption{\label{tab:sc_weights}Synthetic German minority weights}
\centering
\begin{tabular}{lr}
\toprule
Ethnic group & $W$-Weight\\
\midrule
Russian & 0.36\\
Greek & 0.28\\
Finnish & 0.17\\
Lithuanian & 0.08\\
Khakas & 0.05\\
Yakut & 0.05\\
Bulgarian & 0.01\\
\bottomrule
\end{tabular}
\end{table}


Figure \ref{fig_sc_comp_plot} shows the trends in arrests for the German minority and its synthetic control. The synthetic German minority tracks the actual values fairly well except for two large negative shocks to the actual arrests in 1931 and 1932 which the synthetic control does not capture. The trends start to diverge in the second quarter of 1933 with the actual arrests of Germans holding steady but decreasing for synthetic control. 
The gap between the trends for the actual German minority and its synthetic control (shown in figure \ref{fig:sc_placebo_gaps_all}) keeps within the range of 1.25 to 2.5 for most of the post-1933 period. This implies that the rise of Hitler led to about  2\% increase in the arrests of Germans by the NKVD  in the period from 1933 to 1939. This is very similar to the estimates obtained using difference-in-differences. 
%The figure shows increase between 1937 and 1939 (the period of the Great Terror).  
\begin{figure}[h]
\centering
\caption{Comparison plot}
\includegraphics[width=\textwidth]{plots/synthetic_control/until_pact/comparison_plot.pdf}
\label{fig_sc_comp_plot}
\end{figure}


%\subsubsection{Inference}
The synthetic control method, however, by itself does not provide us with any measure of uncertainty of significance. This has been addressed by performing placebo tests \citep{abadie_synthetic_2010}. Synthetic control method is applied iteratively to every ethnicity in the donor pool
%with the same treatment period
as if they were treated. By comparing the gaps from these placebo tests with the gap for the German minority, we can assess the
significance of our results.
Large gaps for arrests of Germans relative to other ethnic groups would suggest that the results are significant since these results would be less likely if there were no treatment effect . 
%The estimated  gaps between the synthetic control and the actual data for every ethnic group is shown in figure \ref{fig:sc_placebo_gaps}. 

Figure \ref{fig:sc_placebo_gaps_all} shows gaps between the synthetic control and the actual trends for Germans together with placebo gaps for all 17 other ethnic groups. The post-treatment gap for German minority is relatively large although not the highest. The figure also highlights that for some ethnic groups the pre-treatment gaps are large too. This indicates that synthetic control of these ethnic groups does not capture the actual pre-treatment trends well. As \citet{abadie_synthetic_2010} note, placebo synthetic controls  with poor pre-treatment fit do not provide good comparison
%are not very helpful for
for estimating rareness of large  post-treatment gap for a treatment with a good pre-treatment fit. They thus recommend excluding excluding placebo groups with substantially higher pre-treatment mean squared prediction error  (MSPE)
%(the average of squared differences between the value of the outcome for synthetic and actual) 
%relatively to the treated group. 
%In our case the pre-treatment MSPE of the German minority is fairly small (0.52). 
%If we exclude minorities with the pre-treatment MSPE
%If we choose the cutoff for exclusion as the MSPE 

%Morever, the pre-treatment MSPE is quite small (0.52) which means that the synthetic German minority captures the actual pre-treatment trends relatively well. However, the pre-treatment MSPE of a few minorities is much larger indicating that no combination of donor ethnic groups fits well their time series. As \citet{abadie_synthetic_2010} note,  if there is poor fit of the synthetic control to the actual trends in the pre-treatment period then its post-treatment gap does not provide good comparison to the well fit ethnic groups. 
%large post-treatment gap would not indicate the presence of an effect but  
Following \citet{abadie_synthetic_2010}  we therefore exclude ethnic groups whose pre-treatment MSPE is 5 times higher then the same measure for German minority. This removes 4 ethnic groups with the worst pre-treatment fit. The resulting plot is shown in the figure \ref{fig:sc_placebo_gaps_all_5_times}. The post-1933 gaps in German arrests now stand out more clearly. 

\begin{figure}[hbtp] 
\caption{Gaps between synthetic control and actual values for placebo tests}
\begin{subfigure}{\textwidth}
\caption{All ethnic groups}
\includegraphics[width=0.9\linewidth]{plots/synthetic_control/ethnicity_imputation/annual/placebo_highlight_all_imp_date.pdf}
\label{fig:sc_placebo_gaps_all}
\end{subfigure}
\begin{subfigure}{\textwidth}
\caption{Ethnic groups with pre-treatment MSPE higher than 5 times the MSPE of Germans excluded}
\includegraphics[width=0.9\linewidth]{plots/synthetic_control/ethnicity_imputation/annual/placebo_highlight_mspe_5_lower_imp_date.pdf}
%Ethnic groups with pre-treatment MSPE higher than 5 times the MSPE of Germany excluded
\label{fig:sc_placebo_gaps_all_5_times}
\end{subfigure}
\label{fig:sc_placebo_gaps}
\end{figure}

Nevertheless, the choice of any level of the cutoff of pre-treatment MSPE %for exclusion of 
is somewhat arbitrary. Alternative way to asses significance of results may be to compare the ratios of  post/pre-treatment MSPE.  The values of these ratios for all ethnic groups are displayed in the figure \ref{fig:sc_mspe_ratios}. Post/pre-treatment MSPE ratio for the German minority is by far the highest. The probability of German minority having the highest ratio of all under the null hypothesis of zero treatment effect is 1/17 ($\approx 0.06$). 

%This is shown in  the figure \ref{fig_sc_mspe_hist}
\begin{figure}[h]
\centering
\includegraphics[width=\textwidth]{plots/synthetic_control/mspe_ratios.pdf}
\caption{Ratios of post-treatment MSPE to pre-treatment MSPE}
\label{fig:sc_mspe_ratios}
\end{figure}
%\subsection{Generalized Synthetic Control Method}
%\citet{xu_generalized_2017}