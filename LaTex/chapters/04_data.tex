Our data on soviet repressions come from  \citet{zhukov_stalins_2018}\footnote{In particular, we downloaded the data from the replications file archive of the journal available  at \url{https://www.prio.org/JPR/Datasets/}} who use  the Victims of Political Terror archive \footnote{The Memorial archive can be accessed at \url{http://base.memo.ru/} (new version) or at \url{http://lists.memo.ru/} (older version)} collected by a Russian NGO Memorial. The main sources of the Memorial lists are declassified Russian Interior Ministry documents, prosecutor’s offices and the Commission for the Rehabilitation of Victims of Political Repression.
 The Memorial archives include 2.6 individual arrest by the Soviet secret police (NKVD) between  the years 1921 and 1959 with names of each person, date of arrest, the place of birth for all observations and  in many cases additional information such as ethnicity, occupation and party membership. 
 However the data are not complete and include about 70\% of estimated 3.8 million convicted for political reasons.

We created our main dataset by counting number of arrest for each ethnicity by year.  A few people who were categorized as having multiple ethnicities were dropped from the dataset and not counted. 
With 17  minorities (Armenian, Belarussian, Estonian, German, Greek, Chechen, Chinese, Jewish, Kabardin, Kalmyk, Korean, Latvian, Lithuanian, Ossetian, Polish, Tatar and Ukrainian) and 37 time periods (from 1921 to 1958) this gives us 663 observations in total. Total number of arrests for each ethnicity is provided in the table \ref{tab:} and the plot of arrest by ethnicity and year (after applying the transformation $\log\left(1 + y_{it}\right)$) is shown in figure \ref{fig:universe}, both in the appendix. 
