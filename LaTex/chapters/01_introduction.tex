What determines the attitude of a state toward ethnic minorities within
its borders? Why are some minorities accommodated or assimilated and
others are politically excluded and repressed? Furthermore,  why does the
position of a state toward its minorities change in time? For example,
Soviet Union largely accommodated its minorities by in 1920s but 
heavily repressed them in the campaigns of  mass terror 10 years later. 

%Some studies emphasize that certain domestic institution such as democracy can decrease the likelihood of persecutions. However this does not explain why the same states often treat its ethnic minorities so differently.  For example (give example, man)...\\
%\citet{mylonas_politics_2013} argues that geopolitical concerns play inportant role. Specifically, the state is likely to choose repression and exclusion against a minority group if a external power with ethnic ties to the group is a geopolitical enemy.  
\citet{mylonas_politics_2013} argues that the geopolitical concerns play an important role. In particular, a state is likely to choose repression and
exclusion if the ethnic minority's country of origin is seen as an
geopolitical enemy. The minority is then viewed by the state as a potential fifth column of the foreign country that could betray the state in case of conflict.   

We test this hypothesis on the case of German minority in Soviet union.
The German-Soviet relations went through a series of fundamental changes in  the first half of the 20th century. 
First, Hitlers rise to power in 1933 changed Germany from a neutral actor to an ideological and geopolitical enemy in the perspective of the Soviet Union. The hostilities ceased in August 1939 with signing of the Molotov-Ribbentrop Pact. But this period did not last long as it was abruptly ended by the German invasion into the USSR in June 1941. 
Our empirical strategy is to compare the change in repressions of Germans throughout these different phases  with the change for other ethnic groups in the Soviet Union using the difference-in-differences design and the synthetic control method. 

The source of the data on repressions is a database of  Russian human rights organization \citet{memorial_zhertvy_2017} which contains more than 2 million of records of individuals arrested by the Soviet secret police obtained mostly from digitized archival materials. 
The challenge with the data is that information on ethnicity and date of arrest is often missing which we address by imputing the missing values using names and date of trial, respectively. 
%We can then see how the repression changed before and after 1933 and compare it with other minorities. 
%In particular, we use the individual arrests by the Soviet secret police as a dependent variable and employ the difference in difference design and the synthetic control method. 


The thesis has the following structure. First, we summarize the existing literature on the topic in section \ref{sec:lit_rev}. Next, section \ref{sec:hist_back} provides necessary historical context on German-Soviet relations, political repressions, and position of ethnic minorities in the USSR. This is followed by section \ref{sec:data} where we describe the sources of the data and provide summary statistics of our  dataset. 
In section \ref{sec:missing_data}, we present the methods that we use to impute missing information on ethnicity and  date of arrest. In section \ref{sec:methodology}, we  describe the two methods that we use to empirically estimate the effect. 
The results of our analysis  are provided in section \ref{sec:results}. Additional robustness checks which asses the sensitivity of results to different specifications are presented in section \ref{subsec:robust_checks}. Finally, we discuss implications of the results in the conclusion. 

%First, the Soviet Union was large multiethnic state whose attitude to its minorities drastically changed throughout the year with   

%We use the case of German minority in the Soviet Union to test this hypothesis for several reason.  

%We test hypothesis put forward by Mylonas (2012) according to which the host state is likely to choose repression and exclusion if the ethnic minority's country of origin is seen as geopolitical enemy. 


%We test this hypothesis on the case of German minority in Soviet using the rise of Hitler as a change of geopolitical relations. 
